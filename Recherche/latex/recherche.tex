\documentclass[12pt,a4paper]{article}

% --------------------
% Pakete
% --------------------
\usepackage[ngerman]{babel}
\usepackage[T1]{fontenc}
\usepackage[utf8]{inputenc}
\usepackage{lmodern}
\usepackage{geometry}
\usepackage{setspace}
\usepackage{graphicx}
\usepackage{amsmath}
\usepackage{hyperref}
\usepackage{csquotes}
\usepackage{biblatex}

\geometry{left=3cm,right=3cm,top=2.5cm,bottom=2.5cm}
\onehalfspacing

\addbibresource{literatur.bib} % Literaturdatei

% --------------------
% Dokument
% --------------------
\begin{document}

% --------------------
% Titelseite
% --------------------
\begin{titlepage}
    \centering
    \vspace*{2cm}
    
    {\Large \textbf{Recherche - LineXpress}}\\[1.5cm]
    
    \textbf{Team:} 3\\
    \textbf{Mitglieder:} Sven Pfister, Mathias Menchini, Timo Kradolfer, Patrick Itten, Joel Rechsteiner, Nils Kreienbühl\\
    \textbf{Studiengang:} Systemtechnik\\
    \textbf{Modul:} PM2\\
    \textbf{Dozent/in:} Michale Wüthrich\\[1cm]
    
    \textbf{Datum:} \today
    
    \vfill
\end{titlepage}

% --------------------
% Inhaltsverzeichnis
% --------------------
\tableofcontents
\newpage

% --------------------
% Einleitung
% --------------------
\section{Einleitung}

\subsubsection{Ausgangslage}
Ausgangslage gemäss Aufgabenstellung - Später noch weiter ausführen, damit schön aussieht.

\subsubsection{Unterteilung}
Jedes mechatronische System kann in drei Hauptsegmente unterteilt werden: Mechanik, Elektronik und Informatik.

% Beispiel für eine Abbildung
\begin{figure}[h]
    \centering
    \includegraphics[width=0.6\textwidth]{img/Mechatronic.png}
    \caption{Mechatronic - Quelle Wikipedia}
    \label{fig:beispiel}
\end{figure}

In dieser Recherche werden für jedes dieser Segmente mögliche Optionen, Bauteile und Lösungsansätze untersucht. Dementsprechend ist dieses Dokument in drei Kapitel gegliedert.

Die Recherche dient anschließend als Grundlage zur Entwicklung weiterer Konzepte, beispielsweise in Form eines morphologischen Kastens.

Da wir mit Rapid Prototyping arbeiten, bleibt die Recherche dynamisch. Dieses Dokument wird fortlaufend weiterentwickelt, und neue Ideen sowie gewonnene Erkenntnisse fließen kontinuierlich ein.
% --------------------
% Hauptteil
% --------------------
\section{Mechanik}

Darstellung wichtiger Begriffe und Konzepte.

\subsection{Fahrwerk}

\subsubsection{Raupen}

\subsubsection{Zweiachsig mit zwei DC-Motoren}

\subsubsection{Zweiachsig mit Servo-Lenkung und DC-Motor}

\subsection{Fahrgestelle}

\subsection{Greifmechanismen}

\subsection{Lagermechanismen}

\subsubsection{Lagertrommel}

\subsubsection{FILO-Stack}

\section{Elektronik}

\subsubsection{NUCLEO-F446RE}

\subsubsection{ZHAW PES Board}

\subsubsection{Eletronische Bauteile}

\paragraph{Aktorik}

\subparagraph{DC-Motor}

\subparagraph{Servo-Motor}
dreht dank strom

\subparagraph{LED}

\paragraph{Sensorik}

\subparagraph{IR-Sensor}

\subparagraph{Ultrscvhall-Sensor}

\subsubsection{Drucksensor}

\subsubsection{Line-Follower-Array}


\section{Informatik}



% Beispiel für eine Abbildung
%\begin{figure}[h]
%    \centering
%    \includegraphics[width=0.6\textwidth]{beispielbild.png}
%    \caption{Beispielabbildung}
%    \label{fig:beispiel}
%\end{figure}

% --------------------
% Fazit
% --------------------
\section{Vergleichbare Projekte}

Zusammenfassung der wichtigsten Ergebnisse und Ausblick.

% --------------------
% Literaturverzeichnis
% --------------------
\newpage
\printbibliography

\end{document}
